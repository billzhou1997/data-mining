\documentclass{article}


    \usepackage{listings}
    \usepackage{graphicx} % Used to insert images
    \usepackage{adjustbox} % Used to constrain images to a maximum size
    \usepackage{color} % Allow colors to be defined
    \usepackage{enumerate} % Needed for markdown enumerations to work
    \usepackage{geometry} % Used to adjust the document margins
    \usepackage{amsmath} % Equations
    \usepackage{amssymb} % Equations
    \usepackage[mathletters]{ucs} % Extended unicode (utf-8) support
    \usepackage[utf8x]{inputenc} % Allow utf-8 characters in the tex document
    \usepackage{fancyvrb} % verbatim replacement that allows latex
    \usepackage{grffile} % extends the file name processing of package graphics
                         % to support a larger range
    % The hyperref package gives us a pdf with properly built
    % internal navigation ('pdf bookmarks' for the table of contents,
    % internal cross-reference links, web links for URLs, etc.)
    \usepackage{hyperref}
    \usepackage{longtable} % longtable support required by pandoc >1.10
    \usepackage{booktabs}  % table support for pandoc > 1.12.2

    \usepackage{amsmath,amsthm,amssymb,graphicx,fancyhdr,enumerate,geometry,bm}
    \usepackage{listings, alltt}
    \usepackage{color}
	\usepackage{mdframed}
	\usepackage{verbatimbox}

	% algorithm
	\usepackage{algorithm,algpseudocode}

    \geometry{left=4cm, right=3.5cm, top=3.5cm, bottom=3.5cm}
    \setlength{\parindent}{0pt}
    \def\printsol{}

	% A box for the answers
	\newcommand{\answerbox}[1]{
		\begin{center}
			\fbox{\begin{minipage}{\textwidth}
	        {\color{blue}
		\ifdefined\printsol #1
		\else
		\phantom{
		\begin{minipage}{\textwidth} #1
		\end{minipage}
		}
		\fi}
			\end{minipage}}
		\end{center}
}

    \pagestyle{fancy}
    \fancyhf{}
    \lhead{ORIE 4740/5740, Spring '22, Lab 3}
    \rhead{\thepage}

    \usepackage{amsmath,amssymb}
    \DeclareMathOperator{\E}{\mathbb{E}}

    \begin{document}

        \begin{center}
                {\Large \bf Lab 3: Cross Validation}

                    \bigskip

                        {\bf First Name}: \underline{\hspace{2.5cm}}  {\bf Last Name}: \underline{\hspace{2.5cm}} {\bf NetID}: \underline{\hspace{2.5cm}}
                    \end{center}
\textbf{Lab 3 is due Feb 25th (Friday) by 11:59pm on Gradescope. Submit your answers
as one .pdf file, including all R code that you had to write to obtain your
answers. Regrade requests should be submitted via Gradescope.}

\bigskip

In this lab, we will learn to apply Cross-Validation techniques to select the
degrees of polynomials in regression. We will consider the validation set
approach, LOOCV and k-fold CV, and apply them to the ToyotaCorolla data set.
Some of the commands in this lab may take a while to run in your computer.
\bigskip

Download the \texttt{ToyotaCorolla.csv} data set from Canvas.  Our goal is to
use a linear regression model with high order terms of different degrees to
predict the price of a vehicle for resell in the market using the variables Age
and KM, and then find the appropriate order of the polynomial of these variables for
our model.
\bigskip

Before analyzing the data, do the same procedures to clean it as we did previously, i.e.
\begin{enumerate}
	\item Delete all other columns except for \texttt{Age\_08\_04} and \texttt{KM}, which we will be using.
	\item Check if there is any missing data. If so, deal with it.
	\item Check the classes of the variables.
\end{enumerate}

\subsection*{The Validation Set Approach}
When we think we have enough samples in our dataset, we can split the data into
training / validation / test set. Recall that the validation set is ``revealed''
to our evaluation pipeline, while the test set is only used at the very end
(often as a sanity check for our evaluation procedure). Here, we pick half of
the observations to be our training data. Moreover, we set the random seed to
$1$ for reproducibility.
\begin{Verbatim}[frame=single]
> set.seed(1)
> train_id = sample(nrow(corollas), nrow(corollas)/2)
\end{Verbatim}

Then we fit a linear regression model using only the data points in the training set.
\begin{Verbatim}[frame=single]
> lm.fit = lm(Price~Age_08_04+KM, data=corollas, subset=train_id)
\end{Verbatim}

Now we can calculate the MSE over the validation set. Write the code below to do that.
\begin{Verbatim}[frame=single]


\end{Verbatim}

Our next step is to fit models with high-order terms of the
predictors, making use of the \texttt{poly()} function. Write a function
\texttt{eval\_lm()}, accepting an argument \texttt{deg},
which fits a model with terms of degree \texttt{deg} of the predictors, and
returns the MSE on the complement of the training set.


\begin{Verbatim}[frame=single]








\end{Verbatim}


Calculate the MSE for regressions models with degree up to 5. Note, this may take a while for R to run. What do you
observe?


\begin{Verbatim}[frame=single]









\end{Verbatim}

\bigskip

\subsection*{Leave-One-Out Cross-Validation}

We then consider Leave-One-Out Cross-Validation (LOOCV). The LOOCV estimate can
be automatically computed for any generalized linear model using the
\texttt{glm()} and \texttt{cv.glm()} functions. As previously, we fit and
validate models of degree up to 5.
\begin{Verbatim}[frame=single]
> cv.error=rep(0,5)
> for (i in 1:5) {
+   glm.fit = glm(Price~poly(Age_08_04, i)+poly(KM,i), data = corollas)
+   cv.error[i] = cv.glm(corollas, glm.fit)$delta[1]
+ }
> plot(cv.error, type="b")
\end{Verbatim}

According to the LOOCV estimates, which model do you recommend? Why?
\begin{Verbatim}[frame=single]


\end{Verbatim}

Check $ p $-values in the summary for the cubic model. Describe your observations and compare them with your previous findings.

\begin{Verbatim}[frame=single]









\end{Verbatim}

\subsection*{$ k $-fold Cross-Validation}
The validation set approach and LOOCV can be thought of a two ``extremes'' of
cross-validation. The method of $k$-fold cross-validation aims to interpolate
between the two. The \texttt{cv.glm()} function can also perform $k$-fold CV
(type \texttt{?cv.glm} for help with the syntax).
We will consider polynomials of degrees one to five, and then plot the
corresponding cross-validation errors for $k \in \{5, 10\}$. Write a script that
computes the cross-validation error for all configurations of parameters, and
produces an appropriate plot, and attach the generated plot.


\begin{Verbatim}[frame=single]



















\end{Verbatim}


Report your findings. Do the cases $K = 5$ and $K = 10$ show different behavior?
How do they compare with LOOCV?

\begin{Verbatim}[frame=single]








\end{Verbatim}

Report and compare the running time of the three CV methods. You can use the
following command to record running time. Which is / should be the fastest, and
why?
\begin{Verbatim}[frame=single]
> ptm = proc.time()
> # INSERT YOUR CODE HERE
> proc.time() - ptm
> # CHECK elapsed
\end{Verbatim}

\begin{Verbatim}[frame=single]









\end{Verbatim}

Without running any \texttt{R} code: do different runs of the validation test
approach produce different estimates? How about LOOCV and $k$-fold CV? Why?

\begin{Verbatim}[frame=single]









\end{Verbatim}

\bigskip

\section*{Take Home Questions}

\subsection*{Part 1. The bias-variance tradeoff}

If we use higher degree polynomials, how does it affect the bias, variance and test error of the model? Explain your answer.
\begin{Verbatim}[frame=single]



\end{Verbatim}

\bigskip
Which unknown quantity are the aforementioned approaches trying to estimate?
\begin{Verbatim}[frame=single]


\end{Verbatim}
\bigskip
Consider a case where the "true model" describing some phenomenon is:
$y=f(x)+\epsilon$, where $f(x) = x$ and $\epsilon$ is the irreducible noise
with
distribution $\epsilon \overset{\mathrm{i.i.d.}}{\sim}\mathcal{N}(0,1)$. Our
training set contains two data
points, $x = -1, x=1$ and corresponding $y$'s generated from the true model
above.
\\
We want to predict $y$ when $x = 2$. What would the expected MSE of our
prediction be if we chose $\hat{f}$ to be the constant function $0$?
\\
Hint:
$\E\left[\left(\hat{f}(x)-y\right)^2\right]=\text{Var}\left(\hat{f}(x)\right)+
  \text{Var}(\epsilon)+\left(\E\left[\hat{f}(x)-f(x)\right]\right)^2$
\begin{Verbatim}[frame=single]









\end{Verbatim}

\bigskip
Let $\mathrm{TE}^*$ be your desired test error threshold. After developing your
model and calculating the training and test errors, you notice that your
training error is much lower than your test error. You also notice that your
training error is lower than $\mathrm{TE}^*$ and your test error is higher than
$\mathrm{TE}^*$. Explain why you are seeing these errors and what steps you
would take to remedy/improve your model.

\begin{Verbatim}[frame=single]







\end{Verbatim}

\bigskip
In this scenario, you notice that your test error and training error are very close to each other but both are still significantly above $TE^*$. Explain why you might see this situation and what steps you would take to remedy/improve your model.

\begin{Verbatim}[frame=single]






\end{Verbatim}
\bigskip

What would happen if the number of features are significantly greater than the
number of observations (i.e. $p \gg n$)? How would the training error and test
error behave?

\begin{Verbatim}[frame=single]











\end{Verbatim}

We will now examine the bias-variance tradeoff in practice using R's
\texttt{cars} dataset.
This can be accessed by typing '\texttt{cars}'. If you see an error, try
importing the \texttt{datasets} package by \texttt{library(datasets)}.

We will develop 3 different models : no - intercept model, intercept model, and
non-linear model. Consider the following procedure:
\begin{enumerate}
	\item Set the seed to 2022 with set.seed(2022) so we can create reproducible results
	for each model
	\item Split the cars data into 2/3 of training data and the rest to test
	data
	\item Create your model (response variable = \texttt{dist}, features =
	\texttt{speed})
	\item Use the test set to predict using your model
	\item Calculate the training MSE
	\item Calculate the test MSE
\end{enumerate}
You will need to write your own R code and submit.

What is the training MSE and test MSE of the no intercept linear regression
model?
\begin{Verbatim}[frame=single]


\end{Verbatim}

What is the training MSE and test MSE of the intercept linear regression model?
\begin{Verbatim}[frame=single]



\end{Verbatim}

What is the training MSE and test MSE of the quadratic (non-linear) model?\\
Hint: use poly() as you did in first part of this lab\\

\begin{Verbatim}[frame=single]


\end{Verbatim}

Repeat the above procedure a few times using a different random seed. What do
you notice about the training errors and test errors of the different models?
Can you explain your observations?
\begin{Verbatim}[frame=single]









\end{Verbatim}

\subsection*{Part 2. Logistic regression}
In this section, we will apply logistic regression to predict the class of
synthetic data with noise, and evaluate our models with the cross validation
approaches, as well as another technique called the bootstrap.

Before anything else, let us run the following code to create a synthetic data
set \texttt{df} and make a plot. Again, let us set the random seed to make sure
our results are reproducible.
\begin{Verbatim}[frame=single]
> set.seed(1)
> n = 1000
> x1 <- runif(n)
> x2 <- runif(n, -2, 1)
> z <- (x1-0.2)*(x1-0.5)*(x1-0.9) * 25 - x2*(x2+1.2)*(x2-0.8) + rnorm(n)/3
> y <- as.integer(z>0)
> plot(x1, x2, col=c("red", "blue")[y+1])
> df <- data.frame(x1,x2,y)
\end{Verbatim}

\paragraph{The bootstrap.} \textit{Questions 4 concerning the bootstrap is optional for this lab.} The bootstrap is another
method for model evaluation,
and aims to approximate the out-of-sample (test) error when applied to
classification. In this section, we compare the bootstrap to $k$-fold cross
validation.

Bootstrap works by sampling \textbf{with replacement} from the
original training dataset (with $n$ samples) to generate more datasets of
size $n$ (this means that a data point may appear more than once in the
generated data). For each of the new datasets (which we shall refer to as the
\textit{bootstrap datasets}), the statistic of interest is recomputed,
and its average over all runs is returned. For estimating the (expected)
prediction error, the simplest pipeline is described in
Algorithm~\ref{alg:vanilla-bootstrap}.
\begin{algorithm}
	\caption{Estimating prediction error via the bootstrap}
	\begin{algorithmic}[1]
		\State \textbf{Input}: dataset $\{(x_i, y_i)\}_{i=1}^n$, loss function
		$\ell$.
		\For{$b = 1, 2, \dots, B$}
			\State generate set $\mathcal{S}_b$ by sampling $n$ items with
			replacement from $\{(x_i, y_i)\}_{i = 1}^n$.
			\State form prediction $\hat{f}^b$ by fitting the logistic
			regression model with training data $\mathcal{S}_b$.
			\State compute
			$\widehat{\mathrm{err}}_b := \frac{1}{n} \sum_{i=1}^n \ell(y_i,
			\hat{f}^b(x_i))$
		\EndFor
		\State \Return $\widehat{\mathrm{err}}_{\mathrm{boot}} :=
		\frac{1}{B} \sum_{b = 1}^B \widehat{\mathrm{err}}_b$
		\label{op:err-bootstrap}
	\end{algorithmic}
	\label{alg:vanilla-bootstrap}
\end{algorithm}

\newpage

\paragraph{Questions on the bootstrap.}
\begin{enumerate}
	\item Do you expect Algorithm~\ref{alg:vanilla-bootstrap} to provide a good
	estimate of the prediction error? Why or why not?

\begin{Verbatim}[frame=single]




\end{Verbatim}


	\item Bootstrapping introduces significant overlap
	between the set of data used for training and the validation set (the
	old training set). Show that each training sample gets picked with
	probability $\approx 0.632$ for a fixed bootstrap dataset, as the number
	of training samples $n \to \infty$. What is the expected number of distinct
	samples used in forming a bootstrap dataset, as $n$ grows large?
	\label{item:bootstrap-optional}
	\begin{Verbatim}[frame=single]











	\end{Verbatim}
	\item Using the \texttt{boot} function from the
	library with the same name,
	apply Algorithm~\ref{alg:vanilla-bootstrap} with $B = 10$ to the logistic
	regression problem, using the data in \texttt{df}.You will need to
	implement a function \texttt{boot\_fn(data, train\_id)} to pass to
	\texttt{boot}, which will do the following:
	\begin{enumerate}[(i)]
		\item using the subset of \texttt{data} designated by the indices
		\texttt{train\_id}, fit a logistic regression using \texttt{glm()}
		(and its \texttt{family} parameters; see the \texttt{R} help text).
		\item return the mean prediction error on the {\bf whole} dataset
		using the \texttt{glm} fit.
	\end{enumerate}

\begin{Verbatim}[frame=single]












\end{Verbatim}

	\item (Optional)
	In an attempt to combine the best of both worlds, the leave-one-out
	bootstrap
	estimate replaces Line~\ref{op:err-bootstrap} with the following estimate:
	\begin{equation}
	\frac{1}{n} \sum_{i=1}^n \frac{1}{|B_{\setminus \{i\}}|}
	\sum_{b \in B_{\setminus \{i\}}} \ell(y_i, \hat{f}^b(x_i)),
	\label{eq:loo-bootstrap}
	\end{equation}
	where $B_{\setminus \{i\}}$ is the set of bootstrap datasets which do not
	contain sample $i$ from the original dataset. Note that we may swap the
	above as
	\begin{equation}
		\frac{1}{n} \sum_{b \in B} \sum_{i \notin \mathcal{S}_b}^n \frac{1}{|
		B_{\setminus \{i\}} |} \cdot \ell(y_i, \hat{f}^b(x_i))
		\label{eq:loo-bootstrap-rewrite}
	\end{equation}
	We will implement Algorithm~\ref{alg:vanilla-bootstrap} with this
	modification step-by-step in the next few questions:
	\begin{enumerate}[(i)]
		\item Write a function $\texttt{boot\_poly}$ that accepts two arguments,
		\texttt{train\_id, deg}, where the first is the training indices and
		the second is the polynomial degree, and fits a logistic regression
		model with \texttt{deg}-polynomial terms for the predictors, using the
		samples corresponding to \texttt{train\_id}, returning the fit model.
		(\textit{Hint: Use the \texttt{glm()} function and is \texttt{family}
		parameter to specify that you are fitting a logistic reg. model.}


\begin{Verbatim}[frame=single]











\end{Verbatim}

		\item Write a function $\texttt{boot\_num}$ that accepts two arguments,
		an index \texttt{i} and a $n \times B$ \textbf{matrix} where column $b$
		contains the training indices of bootstrap $b$, and returns the number
		of bootstrap datasets that do not contain \texttt{i}.

\begin{Verbatim}[frame=single]






\end{Verbatim}


		\item Use the following code to fit $B = 10$ bootstrap datasets for
		given degree \texttt{deg = 2} as a guide:
		\begin{verbatim}
			> B <- 10
			> deg <- 2
			> all_inds <- matrix(sample(1:n, B * n, replace=TRUE), ncol=B)
			> boot_fits <- apply(all_inds, 2,
			+ function(trainind) return(boot_poly(df,trainind, deg)))

		\end{verbatim}
		The above code returns a $B$-dimensional vector of \texttt{glm} objects.
		Using \texttt{bootNum}, write a function
		computing~\eqref{eq:loo-bootstrap-rewrite}.
		(\textit{Hint}: while not necessary, you may find the functions
		\texttt{apply, lapply, sapply} useful.)
		\begin{Verbatim}[frame=single]












		\end{Verbatim}

	\end{enumerate}
\end{enumerate}





\paragraph{Logistic regression with cross-validation.}
We want to fit logistic regression models of different degrees $p$, up to $p =
5$ (see the \texttt{family} parameter of the \texttt{glm()} function).
First, hold out $100$ samples of the dataset as a test set, and use the
remainder as training set. Use $k$-fold cross validation with $k = 10$ and
LOOCV to evaluate the performance for each degree $p$, and compare with LOOCV.
Then, choose the best model from the $k$-fold validation step and compute its
error on the test set. Some remarks:
\begin{itemize}
	\item we are using the overall accuracy rate as the metric for the error.
	\item define a cost function using the following code, and pass that
	function as the cost argument to \texttt{cv.glm()}:
\begin{Verbatim}[frame=single]
	cost_fn <- function(r, pi=0) {
		mean(abs(r - pi) > 0.5)
	}
\end{Verbatim}
	What does the \texttt{cost\_fn} function do, and how does it relate to
	classification with logistic regression?
	\begin{Verbatim}[frame=single]






	\end{Verbatim}

	\item when converting the predictions to integers, which threshold should
	you use?
	\begin{Verbatim}[frame=single]

	\end{Verbatim}
\end{itemize}

Report your findings below:
\begin{enumerate}
	\item What are the validation errors you get for different degrees
	($k$-fold vs. LOOCV)? Which model do they suggest you should choose? Is
	the choice reasonable?
	\begin{Verbatim}[frame=single]






	\end{Verbatim}

	\item What is the test error of your chosen model?
	\begin{Verbatim}[frame=single]




	\end{Verbatim}


\end{enumerate}


\paragraph{Concept map.} Make a concept map for the topics covered by this lab.
While there is no single right answer, you are expected to demonstrate that you
understand the relationships between various topics in the lab.

\end{document}
